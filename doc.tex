\documentclass{article}
\usepackage[brazil, brazilian]{babel}
\usepackage{amsmath, amssymb, amsfonts, amsthm}
\usepackage{xcolor}

\title{Documentação}
\author{Gustavo Pauzner Mezzovilla}

\begin{document}
\maketitle

Funções implementadas no momento:

$$
\begin{array}{|c|l|}
\hline \text {I}s & \text { Insere a string } s \text { na posição atual do texto } \\
\color{red}\texttt {A}n & \color{red}\text { Carrega o conteúdo do arquivo de texto de nome } n \text { no editor } \\
\texttt{E}n & \text { (Sobre)escreve o conteúdo do editor no arquivo de texto de nome } n \\
\texttt{ F } & \text { Move o cursor um caractere à frente } \\
\texttt{ T } & \text { Move o cursor um caractere para trás } \\
\texttt{O} & \text { Move o cursor para o início da linha atual } \\
\texttt{P} & \text { Move cursor para início da próxima palavra (dentro da mesma linha) } \\
\texttt{ Q } & \text { Move cursor para início da palavra atual } \\
\$ & \text { Move o cursor para o fim da linha atual } \\
\texttt{:}x & \text { Move o cursor para o início da linha } x \\
\texttt{:F} & \text { Move o cursor para a última linha do arquivo } \\
\texttt{D} & \text { Apaga o caractere da posição atual } \\
\color{red}\texttt{DW} & \color{red}\text { Apaga a palavra em que o cursor se encontra } \\
\color{red}\texttt{DL} & \color{red}\text { Deleta a linha atual } \\
%\text {M} & \text { Marca (lembra) a posição atual do cursor } \\
\texttt{ V } & \text { Desempilha e insere o conteúdo do topo pilha na posição atual } \\
\texttt{ C } & \text { Empilha o texto entre a posição marcada e a posição atual (sem modificá-lo) } \\
\texttt{ X } & \text { Empilha o texto entre a posição marcada e a posição atual e o deleta } \\
\texttt{ B}s & \text { Busca pela próxima ocorrência do padrão } s \text { no texto } \\
\color{red}\texttt{ S}s/r & \color{red}\text { Substitui toda ocorrência de } s \text { por } r \text { no texto a partir da posição atual } \\
\color{blue}\texttt{ N } & \color{blue}\text { Separa linha atual na posição do cursor } \\
\color{red}\texttt{ U } & \color{red}\text { Unir linha atual e a próxima } \\
\texttt{ ! } & \text { Encerra o programa } \\
\texttt{ J } & \text { Ir para próxima linha (manter a mesma coluna, se possível) } \\
\texttt{ H } & \text { Ir para a linha anterior (manter a mesma coluna, se possível) } \\
%\text { Z } & \text { Exibe a pilha de memória, começando pelo topo. } \\
\hline
\end{array}
$$

As funções em vermelho ainda não foram implementadas ou estão com alguns problemas. As funções em azul estão implementadas porém precisam de atualização.


\end{document}